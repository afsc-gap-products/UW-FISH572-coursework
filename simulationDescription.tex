% Options for packages loaded elsewhere
\PassOptionsToPackage{unicode}{hyperref}
\PassOptionsToPackage{hyphens}{url}
\documentclass[
]{article}
\usepackage{xcolor}
\usepackage[margin=1in]{geometry}
\usepackage{amsmath,amssymb}
\setcounter{secnumdepth}{-\maxdimen} % remove section numbering
\usepackage{iftex}
\ifPDFTeX
  \usepackage[T1]{fontenc}
  \usepackage[utf8]{inputenc}
  \usepackage{textcomp} % provide euro and other symbols
\else % if luatex or xetex
  \usepackage{unicode-math} % this also loads fontspec
  \defaultfontfeatures{Scale=MatchLowercase}
  \defaultfontfeatures[\rmfamily]{Ligatures=TeX,Scale=1}
\fi
\usepackage{lmodern}
\ifPDFTeX\else
  % xetex/luatex font selection
\fi
% Use upquote if available, for straight quotes in verbatim environments
\IfFileExists{upquote.sty}{\usepackage{upquote}}{}
\IfFileExists{microtype.sty}{% use microtype if available
  \usepackage[]{microtype}
  \UseMicrotypeSet[protrusion]{basicmath} % disable protrusion for tt fonts
}{}
\makeatletter
\@ifundefined{KOMAClassName}{% if non-KOMA class
  \IfFileExists{parskip.sty}{%
    \usepackage{parskip}
  }{% else
    \setlength{\parindent}{0pt}
    \setlength{\parskip}{6pt plus 2pt minus 1pt}}
}{% if KOMA class
  \KOMAoptions{parskip=half}}
\makeatother
\usepackage{color}
\usepackage{fancyvrb}
\newcommand{\VerbBar}{|}
\newcommand{\VERB}{\Verb[commandchars=\\\{\}]}
\DefineVerbatimEnvironment{Highlighting}{Verbatim}{commandchars=\\\{\}}
% Add ',fontsize=\small' for more characters per line
\usepackage{framed}
\definecolor{shadecolor}{RGB}{248,248,248}
\newenvironment{Shaded}{\begin{snugshade}}{\end{snugshade}}
\newcommand{\AlertTok}[1]{\textcolor[rgb]{0.94,0.16,0.16}{#1}}
\newcommand{\AnnotationTok}[1]{\textcolor[rgb]{0.56,0.35,0.01}{\textbf{\textit{#1}}}}
\newcommand{\AttributeTok}[1]{\textcolor[rgb]{0.13,0.29,0.53}{#1}}
\newcommand{\BaseNTok}[1]{\textcolor[rgb]{0.00,0.00,0.81}{#1}}
\newcommand{\BuiltInTok}[1]{#1}
\newcommand{\CharTok}[1]{\textcolor[rgb]{0.31,0.60,0.02}{#1}}
\newcommand{\CommentTok}[1]{\textcolor[rgb]{0.56,0.35,0.01}{\textit{#1}}}
\newcommand{\CommentVarTok}[1]{\textcolor[rgb]{0.56,0.35,0.01}{\textbf{\textit{#1}}}}
\newcommand{\ConstantTok}[1]{\textcolor[rgb]{0.56,0.35,0.01}{#1}}
\newcommand{\ControlFlowTok}[1]{\textcolor[rgb]{0.13,0.29,0.53}{\textbf{#1}}}
\newcommand{\DataTypeTok}[1]{\textcolor[rgb]{0.13,0.29,0.53}{#1}}
\newcommand{\DecValTok}[1]{\textcolor[rgb]{0.00,0.00,0.81}{#1}}
\newcommand{\DocumentationTok}[1]{\textcolor[rgb]{0.56,0.35,0.01}{\textbf{\textit{#1}}}}
\newcommand{\ErrorTok}[1]{\textcolor[rgb]{0.64,0.00,0.00}{\textbf{#1}}}
\newcommand{\ExtensionTok}[1]{#1}
\newcommand{\FloatTok}[1]{\textcolor[rgb]{0.00,0.00,0.81}{#1}}
\newcommand{\FunctionTok}[1]{\textcolor[rgb]{0.13,0.29,0.53}{\textbf{#1}}}
\newcommand{\ImportTok}[1]{#1}
\newcommand{\InformationTok}[1]{\textcolor[rgb]{0.56,0.35,0.01}{\textbf{\textit{#1}}}}
\newcommand{\KeywordTok}[1]{\textcolor[rgb]{0.13,0.29,0.53}{\textbf{#1}}}
\newcommand{\NormalTok}[1]{#1}
\newcommand{\OperatorTok}[1]{\textcolor[rgb]{0.81,0.36,0.00}{\textbf{#1}}}
\newcommand{\OtherTok}[1]{\textcolor[rgb]{0.56,0.35,0.01}{#1}}
\newcommand{\PreprocessorTok}[1]{\textcolor[rgb]{0.56,0.35,0.01}{\textit{#1}}}
\newcommand{\RegionMarkerTok}[1]{#1}
\newcommand{\SpecialCharTok}[1]{\textcolor[rgb]{0.81,0.36,0.00}{\textbf{#1}}}
\newcommand{\SpecialStringTok}[1]{\textcolor[rgb]{0.31,0.60,0.02}{#1}}
\newcommand{\StringTok}[1]{\textcolor[rgb]{0.31,0.60,0.02}{#1}}
\newcommand{\VariableTok}[1]{\textcolor[rgb]{0.00,0.00,0.00}{#1}}
\newcommand{\VerbatimStringTok}[1]{\textcolor[rgb]{0.31,0.60,0.02}{#1}}
\newcommand{\WarningTok}[1]{\textcolor[rgb]{0.56,0.35,0.01}{\textbf{\textit{#1}}}}
\usepackage{graphicx}
\makeatletter
\newsavebox\pandoc@box
\newcommand*\pandocbounded[1]{% scales image to fit in text height/width
  \sbox\pandoc@box{#1}%
  \Gscale@div\@tempa{\textheight}{\dimexpr\ht\pandoc@box+\dp\pandoc@box\relax}%
  \Gscale@div\@tempb{\linewidth}{\wd\pandoc@box}%
  \ifdim\@tempb\p@<\@tempa\p@\let\@tempa\@tempb\fi% select the smaller of both
  \ifdim\@tempa\p@<\p@\scalebox{\@tempa}{\usebox\pandoc@box}%
  \else\usebox{\pandoc@box}%
  \fi%
}
% Set default figure placement to htbp
\def\fps@figure{htbp}
\makeatother
\setlength{\emergencystretch}{3em} % prevent overfull lines
\providecommand{\tightlist}{%
  \setlength{\itemsep}{0pt}\setlength{\parskip}{0pt}}
\usepackage{bookmark}
\IfFileExists{xurl.sty}{\usepackage{xurl}}{} % add URL line breaks if available
\urlstyle{same}
\hypersetup{
  pdftitle={Using simulated data for FISH 572},
  pdfauthor={FISH 572, Winter 2026},
  hidelinks,
  pdfcreator={LaTeX via pandoc}}

\title{Using simulated data for FISH 572}
\author{FISH 572, Winter 2026}
\date{2025-11-14}

\begin{document}
\maketitle

\section{Introduction}\label{introduction}

The mid-course project for Fish 572 is to analyze survey data, which may
be a data set of their choice or simulated data provided by the
instructors. This document describes how to use one of the simulated
datasets. The methods described here may be extended to a real survey
dataset.

\section{Simulated data}\label{simulated-data}

Data were simulated across a square grid of 100 X 100 cells with depth
contours. Here is a plot of the locations on the grid with deeper depths
shown in darker blue.

\begin{Shaded}
\begin{Highlighting}[]
\CommentTok{\# make sure in the correct directory}
\NormalTok{dat }\OtherTok{\textless{}{-}} \FunctionTok{readRDS}\NormalTok{(}\StringTok{"coursework/simulations/sim\_data/sim\_dat\_1.RDS"}\NormalTok{)}
\NormalTok{predictor\_dat }\OtherTok{\textless{}{-}} \FunctionTok{readRDS}\NormalTok{(}\StringTok{"coursework/simulations/sim\_data/grid.RDS"}\NormalTok{)}
\NormalTok{predictor\_dat}\SpecialCharTok{$}\NormalTok{negDepth }\OtherTok{\textless{}{-}} \SpecialCharTok{{-}}\DecValTok{1} \SpecialCharTok{*}\NormalTok{ predictor\_dat}\SpecialCharTok{$}\NormalTok{depth}

\FunctionTok{ggplot}\NormalTok{(dplyr}\SpecialCharTok{::}\FunctionTok{filter}\NormalTok{(predictor\_dat, year }\SpecialCharTok{==} \DecValTok{1}\NormalTok{), }\FunctionTok{aes}\NormalTok{(X, Y)) }\SpecialCharTok{+}
  \FunctionTok{geom\_tile}\NormalTok{(}\FunctionTok{aes}\NormalTok{(}\AttributeTok{fill =}\NormalTok{ negDepth),}\AttributeTok{colour=}\FunctionTok{rgb}\NormalTok{(}\DecValTok{0}\NormalTok{,}\DecValTok{0}\NormalTok{,}\DecValTok{0}\NormalTok{,}\FloatTok{0.1}\NormalTok{),}\AttributeTok{linewidth=}\FloatTok{0.2}\NormalTok{) }\SpecialCharTok{+}
  \FunctionTok{scale\_color\_gradient2}\NormalTok{() }\SpecialCharTok{+} 
  \FunctionTok{scale\_x\_continuous}\NormalTok{(}\AttributeTok{expand =} \FunctionTok{c}\NormalTok{(}\DecValTok{0}\NormalTok{, }\DecValTok{0}\NormalTok{)) }\SpecialCharTok{+} \CommentTok{\# Remove expansion on x{-}axis}
  \FunctionTok{scale\_y\_continuous}\NormalTok{(}\AttributeTok{expand =} \FunctionTok{c}\NormalTok{(}\DecValTok{0}\NormalTok{, }\DecValTok{0}\NormalTok{)) }\CommentTok{\# Remove expansion on x{-}axis}
\end{Highlighting}
\end{Shaded}

\pandocbounded{\includegraphics[keepaspectratio]{simulationDescription_files/figure-latex/grid-1.pdf}}

There are eight (8) columns in a simulated dataset: - year: the year,
being 1:6; - X: the location along the x-axis; - Y: the location along
the y-axis; - eta: the simulated values; - observed: the observed value
of the simulated data (i.e.~observation error applied to eta); -
eta\_scaled: the simulated data scaled by the mean of simulated data in
a reference simulated dataset; - observed\_scaled: the observed data
scaled by the mean of observations in a reference simulated dataset; -
depth\_scaled: the depth scaled by subtracting the mean and dividing by
the standard deviation.

\begin{Shaded}
\begin{Highlighting}[]
\CommentTok{\#plot eta}
\FunctionTok{ggplot}\NormalTok{(dplyr}\SpecialCharTok{::}\FunctionTok{filter}\NormalTok{(dat, year }\SpecialCharTok{==} \DecValTok{1}\NormalTok{), }\FunctionTok{aes}\NormalTok{(X, Y)) }\SpecialCharTok{+}
  \FunctionTok{geom\_tile}\NormalTok{(}\FunctionTok{aes}\NormalTok{(}\AttributeTok{fill =}\NormalTok{ eta),}\AttributeTok{colour=}\FunctionTok{rgb}\NormalTok{(}\DecValTok{1}\NormalTok{,}\DecValTok{1}\NormalTok{,}\DecValTok{1}\NormalTok{,}\FloatTok{0.05}\NormalTok{),}\AttributeTok{linewidth=}\FloatTok{0.2}\NormalTok{) }\SpecialCharTok{+}
  \FunctionTok{scale\_color\_gradient2}\NormalTok{() }\SpecialCharTok{+} 
  \FunctionTok{scale\_x\_continuous}\NormalTok{(}\AttributeTok{expand =} \FunctionTok{c}\NormalTok{(}\DecValTok{0}\NormalTok{, }\DecValTok{0}\NormalTok{)) }\SpecialCharTok{+} \CommentTok{\# Remove expansion on x{-}axis}
  \FunctionTok{scale\_y\_continuous}\NormalTok{(}\AttributeTok{expand =} \FunctionTok{c}\NormalTok{(}\DecValTok{0}\NormalTok{, }\DecValTok{0}\NormalTok{)) }\CommentTok{\# Remove expansion on x{-}axis}
\end{Highlighting}
\end{Shaded}

\pandocbounded{\includegraphics[keepaspectratio]{simulationDescription_files/figure-latex/simulated-1.pdf}}

\begin{Shaded}
\begin{Highlighting}[]
\CommentTok{\#plot observed}
\FunctionTok{ggplot}\NormalTok{(dplyr}\SpecialCharTok{::}\FunctionTok{filter}\NormalTok{(dat, year }\SpecialCharTok{==} \DecValTok{1}\NormalTok{), }\FunctionTok{aes}\NormalTok{(X, Y)) }\SpecialCharTok{+}
  \FunctionTok{geom\_tile}\NormalTok{(}\FunctionTok{aes}\NormalTok{(}\AttributeTok{fill =}\NormalTok{ observed),}\AttributeTok{colour=}\FunctionTok{rgb}\NormalTok{(}\DecValTok{1}\NormalTok{,}\DecValTok{1}\NormalTok{,}\DecValTok{1}\NormalTok{,}\FloatTok{0.05}\NormalTok{),}\AttributeTok{linewidth=}\FloatTok{0.2}\NormalTok{) }\SpecialCharTok{+}
  \FunctionTok{scale\_color\_gradient2}\NormalTok{() }\SpecialCharTok{+} 
  \FunctionTok{scale\_x\_continuous}\NormalTok{(}\AttributeTok{expand =} \FunctionTok{c}\NormalTok{(}\DecValTok{0}\NormalTok{, }\DecValTok{0}\NormalTok{)) }\SpecialCharTok{+} \CommentTok{\# Remove expansion on x{-}axis}
  \FunctionTok{scale\_y\_continuous}\NormalTok{(}\AttributeTok{expand =} \FunctionTok{c}\NormalTok{(}\DecValTok{0}\NormalTok{, }\DecValTok{0}\NormalTok{)) }\CommentTok{\# Remove expansion on x{-}axis}
\end{Highlighting}
\end{Shaded}

\pandocbounded{\includegraphics[keepaspectratio]{simulationDescription_files/figure-latex/observed-1.pdf}}

\section{Sampling from the simulated
data}\label{sampling-from-the-simulated-data}

One potential use of simulated data is to examine alternative sampling
designs. In this example, we sample from the 100 X 100 grid in year 1
using simple random sampling (SRS).

\begin{Shaded}
\begin{Highlighting}[]
\FunctionTok{source}\NormalTok{(}\StringTok{"coursework/simulations/functions\_simulation.R"}\NormalTok{)}
\FunctionTok{set.seed}\NormalTok{(}\DecValTok{245}\NormalTok{)}
\NormalTok{n }\OtherTok{\textless{}{-}} \DecValTok{1500}
\NormalTok{year }\OtherTok{\textless{}{-}} \DecValTok{1}
\NormalTok{dat.yr }\OtherTok{\textless{}{-}}\NormalTok{ dplyr}\SpecialCharTok{::}\FunctionTok{filter}\NormalTok{(dat, year }\SpecialCharTok{==} \DecValTok{1}\NormalTok{)}
\NormalTok{sampleRowNum }\OtherTok{\textless{}{-}} \FunctionTok{sample}\NormalTok{(}\DecValTok{1}\SpecialCharTok{:}\FunctionTok{nrow}\NormalTok{(dat.yr),n,}\AttributeTok{replace=}\NormalTok{F)}
\NormalTok{sampleDat }\OtherTok{\textless{}{-}}\NormalTok{ dat.yr[sampleRowNum,]}
\CommentTok{\# create sample (as if you were a boat sampling once in each sampled grid cell)}
\NormalTok{obsCV }\OtherTok{\textless{}{-}} \FloatTok{0.2}  \CommentTok{\# if using "observed" set this to zero (observation error already applied)}
\NormalTok{catchabilityPars }\OtherTok{\textless{}{-}} \FunctionTok{list}\NormalTok{(}\AttributeTok{mean=}\DecValTok{1}\NormalTok{,}\AttributeTok{var=}\FloatTok{0.1}\NormalTok{) }\CommentTok{\#related to gear selectivity (and availability to net)}
\NormalTok{theSample }\OtherTok{\textless{}{-}} \FunctionTok{sampleGrid.fn}\NormalTok{(sampleDat, obsCV, catchabilityPars,}\AttributeTok{varName=}\StringTok{"eta"}\NormalTok{)}


\CommentTok{\#plot sample}
\FunctionTok{ggplot}\NormalTok{(theSample, }\FunctionTok{aes}\NormalTok{(X, Y)) }\SpecialCharTok{+}
  \FunctionTok{geom\_tile}\NormalTok{(}\FunctionTok{aes}\NormalTok{(}\AttributeTok{fill =}\NormalTok{ observation)) }\SpecialCharTok{+}
  \FunctionTok{scale\_color\_gradient2}\NormalTok{() }\SpecialCharTok{+} 
  \FunctionTok{scale\_x\_continuous}\NormalTok{(}\AttributeTok{expand =} \FunctionTok{c}\NormalTok{(}\DecValTok{0}\NormalTok{, }\DecValTok{0}\NormalTok{)) }\SpecialCharTok{+} \CommentTok{\# Remove expansion on x{-}axis}
  \FunctionTok{scale\_y\_continuous}\NormalTok{(}\AttributeTok{expand =} \FunctionTok{c}\NormalTok{(}\DecValTok{0}\NormalTok{, }\DecValTok{0}\NormalTok{)) }\CommentTok{\# Remove expansion on x{-}axis}
\end{Highlighting}
\end{Shaded}

\pandocbounded{\includegraphics[keepaspectratio]{simulationDescription_files/figure-latex/SRS-1.pdf}}

\section{Analysis}\label{analysis}

Now that you have your samples, the goal is to estimate the population
in all grid cells across the entire sampling area (100 X 100 grid for
the simulated data). In other words, you want to create the simulated
map but you have not sampled all cells, and those that you did sample
are subject to uncertainty. There are numerous ways to analyze the data,
which can be lumped into
\texttt{design-based\textquotesingle{}\textquotesingle{}\ and}modelled'\,'.

\end{document}
